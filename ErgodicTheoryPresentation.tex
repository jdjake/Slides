\documentclass{beamer}
\usetheme{Berlin}

\usepackage{xmpmulti}
\usepackage{graphicx}
\usepackage{verbatim}

\title{On Molecular Gases and the Natural Numbers}
\subtitle{An Introduction to Ergodic Theory}
\author{Jacob Denson}
\institute{University of Alberta}
\date{\today}

\begin{document}

\begin{frame}
\titlepage
\end{frame}

\begin{comment}
\begin{frame}
\frametitle{The n-body problem}
\begin{columns}
\begin{column}{.6\textwidth}
\begin{itemize}
    \item<1-> Given a system of $n$ bodies (think of them as planets), solve the equation governing their motion.
    \item<2-> 1877 - Poincare/Bruns proved that 3-body solution is impossible.
    \item<3-> Can't Approximate because of Chaos.
    \item<4-> Uh Oh...
\end{itemize}
\end{column}
\begin{column}{.4\textwidth}
\includegraphics[width=.8\textwidth]{3body.jpg}
\end{column}
\end{columns}
\end{frame}
\end{comment}

\begin{frame}
\frametitle{Qualitative Understanding of Differential Equations}
\begin{itemize}
    \item<1-> Classic Understanding: Solve equation algebraically
    %
    \[ \frac{d^2 y}{dt^2} = y \frac{dy}{dx} + t \]
    \item<3-> Modern Method: Define the {\it Phase Space} to be the set of initial conditions the equation satisfies. $\mathbf{R}^3$ for equation above.
    \item<4-> The Diffy. Q transforms to a vector field.
    %
    \[ v(t,y,\dot{y}) = (1,\dot{y}, y \dot{y} + t) \]
    \item<5-> The vector field gives us `evolution operators' $f_t$ on phase space, transforming phase space.
\end{itemize}
\end{frame}

\begin{frame}
\frametitle{Let's do an example \dots}
\begin{center}
\onslide<1->{\scalebox{1}{$\ddot{y} = -y$}}\\
\vspace{1em}
\onslide<2->{\scalebox{1}{$v(y,\dot{y}) = (\dot{y}, -y)$}}\\
\vspace{1em}
\onslide<3->{\scalebox{1}{$f_t(x,y) = \begin{pmatrix} \cos t & \sin t \\ -\sin t & \cos t \end{pmatrix} \begin{pmatrix} x \\ y \end{pmatrix}$}}\\
\vspace{1em}
\onslide<4->{\includegraphics[width=.3\textwidth]{springfield.png}}
\end{center}
\end{frame}

\begin{frame}
\frametitle{Ergodic Theory}
\begin{itemize}
    \item<1-> What happens if we look at $f_t$ after a looooong time.
    \item<2-> What happens if we look at the `average value' of $f_t$.
    \item<3-> Since $f_t \circ f_s = f_{t+s}$, let's only look at integer values of $t$, because we get asymptotics, and map is described by a single map $T: X \to X$ iterated (if $T = f_1$, $f_n = T^n$).
    \item<4-> Let's only look at maps which `mix well'.
\end{itemize}
\end{frame}

\begin{frame}
\frametitle{Hamiltonian Mechanics}
\begin{itemize}
    \item<1-> Newton's favourite equation: $m\dot{q} = F(q)$
    \item<2-> Introduce the Potential and Kinetic energies $V$ and $K$, such that
    %
    \[ \nabla V(q) = - F(q)\ \ \ \ \ \ \ \ \ \ K = \frac{m \dot{q}^2}{2} = \frac{p^2}{2m} \]
    %
    And define the total energy
    %
    \[ H(p,q) = V(q) + \frac{p^2}{2m} \]
    \item<3-> New equations defining classical mechanics.
    %
    \[ \dot{p} = - \frac{\partial H}{\partial q}\ \ \ \ \ \ \ \ \ \ \dot{q} = \frac{\partial H}{\partial p} \]
    %
    \item<4-> Corresponding vector field $v$ has zero divergence!
\end{itemize}
\end{frame}

\begin{frame}
\frametitle{Who Cares About Divergence Zero?}
\begin{theorem}
    If a field $v$ has divergence zero, the operators $f_t$ preserve volume.
\end{theorem}
\begin{proof}
    Notice the inequality
    %
    \[ \det(I + M t) = 1 + \text{tr}(M) t + O(t^2) \]
    %
    Apply change of variables formula on volume integral, and differentiate (notice the trace of $Df_t$ is the divergence of $v$).
\end{proof}
%\begin{proof}
%    Write $f_t(x) = x + v(x) t + O(t^2)$. Then $Df_t = I + (Dv) t + O(t^2)$
%    %
%    and for any matrix $M$,
%    %
%    \[ \det(I + M t) = 1 + \text{tr}(M) t + O(t^2) \]
%    %
%    (just expand determinant, and ignore factors with more than one factor of $t$). Notice the trace of $Df_t$ is the divergence. Now fix a subset $U$ of phase space, and define
    %
%    \[ w(t) = \text{vol}(f_tU) = \int_{f_tU} 1 = \int_U |\det(f_t)| \]
    %
%    Finally
    %
%    \[ w'(t) = \frac{d}{dt} \int_U |\det(f_t)| = \int_U \frac{d |\det(f_t)|}{dt} = \int_U \text{div}(v) \]
%\end{proof}
\end{frame}

\begin{frame}
\frametitle{Area Preserving Maps}
\begin{columns}
\begin{column}{.6\textwidth}
\begin{itemize}
    \item<1-> Thus area in `position-momentum' space is preserved.
    \item<2-> Like a Heisenberg uncertainty principle for classical mechanics.
    \item<3-> Area preserving maps prevent space from being `squished' or `minimized'. Cool theorems happen because of this...
\end{itemize}
\end{column}
\begin{column}{.4\textwidth}
\includegraphics[width=\textwidth]{catHamilton.png}
\end{column}
\end{columns}
\end{frame}

\begin{frame}
\frametitle{Recurrence}
\begin{columns}
\begin{column}{.6\textwidth}
\begin{itemize}
    \item<1-> Consider a map $T: X \to X$ which can be seen as moving 1 second into the future $(T = f_1)$.
    \item<2-> Will $x \in X$ ever return to itself? $T^n x = x$?
    \item<3-> What about returning `close' to itself, a neighbourhood of it's original position.
\end{itemize}
\end{column}
\begin{column}{.4\textwidth}
\includegraphics[width=\textwidth]{catHamilton.png}
\end{column}
\end{columns}
\end{frame}

\begin{frame}
\frametitle{Poincare Recurrence Theorem}
\begin{theorem}
    If $T:X \to X$ is an invertible volume-preserving map, where $X$ has bounded problem, and $U \subset X$ is a fixed set of positive volume, then almost every point of $U$ returns to $U$.
\end{theorem}
%\begin{proof}
%    Consider
%    %
%    \[ U, TU, T^2U, T^3U, \dots \]
%    %
%    If $U$ has positive volume, and each $T^iU$ is disjoint, then
%    %
%    \[ \text{vol}(X) \geq \text{vol}\left(\bigcup T^i U \right) = \sum \text{vol}(T^i U) = \sum \text{vol}(U) = \infty \]
%    %
%    Thus $T^i U$ intersects some $T^j U$, but then $U$ intersects $T^{i-j}U$. To see almost every point of $U$ returns to $U$, consider what would happen if the set of non-recurrent points in $U$ has positive volume.
%\end{proof}
\end{frame}

\begin{frame}
\frametitle{Rotations of $S^1$}
\begin{itemize}
    \item<1-> Consider a dynamical system of $S^1$ which rotates the system irrationally (i.e. a map defined by $T(z) = wz$, where $w$ is not a root of unity).
    \item<2-> Poincare: The orbit of a point is dense in $S^1$.
    \item<3-> Stronger Ergodic Theory: The orbit is uniform in $S^1$.
\end{itemize}
\end{frame}

\begin{frame}
\frametitle{Order from Chaos}
\begin{center} \includegraphics[width=\textwidth]{gas.png} \end{center}
\begin{itemize}
    \item<1-> Paradox: Gas Molecules organize themselves over time!
    \item<2-> Bound on time to reccur is $v(X)/v(U)$.
    \item<3-> If willing to get an error of $\varepsilon$, and box has width $m$, then time to reccur is bounded by $(m/\varepsilon)^{3n}$, where $n$ is the particle number. For $n = 6.02 \cdot 10^{23}$, this is a veeeerrrry long time.\\
\end{itemize}
\end{frame}

\begin{frame}
\frametitle{Cat Map}
\begin{itemize}
    \item<1-> Consider the invertible map $T(x,y) = (2x + y, x + y)$, area preserving because it is a linear map of determinant one. Restrict the map to $[0,1]^2$, where addition is considered modulo 1. This map still preserves area, but scrunches up space.
\end{itemize}
\end{frame}

\begin{frame}
\frametitle{Powers of Two}
\begin{itemize}
    \item<1-> Consider the powers of 2: 1,2,4,8,16,32,64,128,256,512,1024,$\dots$, and just take off the first digit.
    %
    \[ 1,2,4,8,1,3,6,1,2,5,1,\dots \]
    %
    \item<2-> No regular pattern, but what about average distribution - is it uniform?
\end{itemize}
\end{frame}

\begin{frame}
\frametitle{Scientific Notation and Benford's law}
\begin{itemize}
    \item<1-> Suppose we write $x$ in scientific notation as $y 10^z$. Then $\log_{10}(x) = z + \log_{10}(y)$, and $\log_{10}(y) < 1$.
    \item<2-> Since $\log$ is increasing, $x$ has first digit $k$ if and only, once we remove the integer part from the number,
    %
    \[ \log_{10}(k) \leq \log_{10}(x) < \log_{10}(k+1) \]
    \item<3-> $\log_{10}(2^n) = n \log_{10}(2)$, so we need only look at shifts along the line by an irrational number to learn about the distribution of $2^n$.
    \item<4-> Poincare Recurrence tells us that the uniform shifts are dense in $[0,1]$. Stronger versions of Poincare recurrence tells us it is evenly distributed, so that the probability of the first digit of a number being $k$ is $\log(k+1) - \log(k)$.
\end{itemize}
\end{frame}

\begin{frame}
\frametitle{There's Lots More!}
\begin{itemize}
    \item<1-> We haven't even described what it means for a Transformation to be Ergodic, or its effects.
    \item<2-> We haven't even gotten to the interesting applications in number theory!
\end{itemize}
\end{frame}

\end{document}